%
 \section{MPI Module Interface}
%
\subsection{Portable Interface to the MPI "mpif.h"}
%
The purpose of \verb"m_mpif" module is to provide a portable
interface of \verb"mpif.h" with different MPI implementation.
By combining module \verb"m_mpif" and \verb"m_mpif90", it may be
possible to build a Fortran 90 MPI binding module graduately.

Although it is possible to use \verb'include "mpif.h"' directly
in individual modules, it has several problems:
\begin{itemize}
\item It may conflict with either the source code of a {\sl fixed}
      format or the code of a {\sl free} format;
\item It does not provide the protection and the safety of using
      these variables as what a \verb"MODULE" would provide.
\end{itemize}

\begin{verbatim}
EXAMPLE 1:
  Some of the m_mpif variable types (and the corresponding mpif.h ones)
  are

    MP_INTEGER          => MPI_INTEGER
    MP_REAL             => MPI_REAL
    MP_DOUBLE_PRECISION => MPI_DOUBLE_PRECISION
    MP_LOGICAL          => MPI_LOGICAL
    MP_CHARACTER        => MPI_CHARACTER
    MP_REAL4            => MPI_REAL4
    MP_REAL8            => MPI_REAL8
\end{verbatim}

More information may be found in the module \verb"m_mpif90".

\subsection{Fortran 90 Style MPI Module Interface}
 %
{\tt m\_mpif90} is a Fortran 90 style MPI module interface.
By wrapping \verb'include "mpif.h"' into a module, \verb"m_mpif()"
provides an easy way to
%
\begin{itemize}
\item Avoid the problem with {\sl fixed} or {\sl free} formatted
      Fortran 90 files;
\item Provide protections with only a limited set of \verb"PUBLIC"
      variables; and
\item Be extended to a MPI Fortran 90 binding.
\end{itemize}
%
Subroutines defined in this module are named with the prefix
{\sl MP\_} corresponding to the {\sl MPI\_} prefix used in MPI
for similar subroutines.
{\tt m\_mpif90} introduces new functions that for instance return
the MPI datatype of a variable.
%
\begin{verbatim}
EXAMPLE 2:
  Code:
      use m_mpif90,only : MP_init
      use m_mpif90,only : MP_finalize
      use m_mpif90,only : MP_comm_world
      use m_mpif90,only : MP_comm_rank

      integer :: ier
      integer :: myID               ! Processor ID
      integer,parameter :: ROOT=0   ! ID of the master processor
      
      call MP_init(ier)
      call MP_comm_rank(MP_comm_world,myID,ier)
      ...
      if (myID==ROOT) then
      ...
      endif
      ...
      call MP_finalize(ier)
\end{verbatim}
%
