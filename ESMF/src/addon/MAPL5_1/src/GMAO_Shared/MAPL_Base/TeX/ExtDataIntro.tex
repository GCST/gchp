{\tt MAPL\_ExtDataGridCompMod} is an internal MAPL gridded component 
used to fulfill imports fields in a MAPL hierarchy from netcdf files
on disk.
It is usually one of the
three gridded components in the ``cap'' or main program of a MAPL application,
the others being the root of the MAPL hierarchy it is servicing and {\tt MAPL\_HistoryGridCompMod}. It is
instanciated and all its registered methods are run automatically by
{\tt MAPL\_Cap,} if that is used.
{\tt MAPL\_ExtDataGridCompMod} will provide
data to fields in the Import states of MAPL components that are not satisfied by a \texttt{MAPL\_AddConnectivity} call
in the MAPL
hierarchy. In a MAPL application fields added to the Import state of a component
are passed up the MAPL hierarchy looking for a connectivity to another component that will
provide data to fill the import. If a connectivity is not found these fields will eventually reach
the ''cap``. At this point any fields that have not have their connectivity satisfied
are passed to the {\tt MAPL\_ExtDataGridCompMod} through its Export state.
{\tt MAPL\_ExtDataGridCompMod} is in essence a provider of last resort for Import fields that
need to be filled with data.

The user provides a resource file available to the {\tt MAPL\_ExtDataGridCompMod} GC.
At its heart this resource file provides a connection between a field name and a variable name in
a netcdf file on disk. The component receives a list of fields that need to be filled
and parses the resource file to determine if {\tt MAPL\_ExtDataGridCompMod} can fill a variable of that name.
We will refer to each field name-file variable combination as a primary export.
Each primary export is an
announcment that {\tt MAPL\_ExtDataGridCompMod} is capable of filling a field named A with data contained in variable B on file xyz.
Note that the field name in each primary export does not need to actually be a field
that needs to be filled by the model. The component only processes primary exports that are needed
The resource file should be viewed as an anncountment of what {\tt MAPL\_ExtDataGridCompMod} can provide.
In addition to simply announcing what {\tt MAPL\_ExtDataGridCompMod} can provide
the user can specify other information such as how frequently to update the data from
disk. This could be at every step, just once when starting the model run, or at a particular
time each days. {\tt MAPL\_ExtDataGridCompMod} also allows data to be shifted and scaled.

{\tt MAPL\_ExtDataGridCompMod} uses {\tt MAPL\_CFIO} to perform the actual file IO and thus the
IO capabilities are limited to files {\tt MAPL\_CFIO} is capable of reading.

{\tt MAPL\_ExtDataGridCompMod} provides an additional method to fill fields. The user can specify
any number of derived exports. Each derived export should once again be viewed as an announcement that
{\tt MAPL\_ExtDataGridCompMod} has the capability to fill a field with a given name.
Instead of the data coming from disk now the user provides a mathematical expression used to evaluate the field
which could include fields from the primary exports.
The function is any character string that obeys
the rules of the {\tt MAPL\_NewArthParserMod}. Any primary export can be used as a variable in the expression,
regardless of whether
that particular primary export is needed to fulfill an import. Care must be taken however
to ensure that the fields are conformal (i.e. on the same grid). For example each field in the expression must have the
same number of levels. Likewise attempting to fill a 2D field from a 3D field is nonsensical
(see the description of {\tt MAPL\_NewArthParserMod} for an exception). 
Any derived exports not needed to fulfill a field are ignored when processing the resource file.
As a simple example suppose we have two primary exports with field names A and B. We could
define a derived export C that is the sum of A and B.

Finally the component provides the user the capability of creating masks based on a Lat-Lon bounding box.
The resulting mask is a real value inside the box and another outside the box. The mask is genereated
when the component is initialized and my be used in subsequent derived expressions.

The ExtData.rc file has the following structure. Each entry for the primary exports, derived exports, or mask entry with space separated arguements.
\begin{verbatim}
PrimaryExports:
field_name Units field_dims level_type climatology 
    refresh_template shift scale Name_On_File FileTemplate
    reference_time frequency_units frequency
::
Masks::
mask_name mask_function
::
# the only currently supported mask function is the bounding box called as
# bbox(field_name,lat_min,lat_max,lon_min,lon_max,real_in,real_out)
# field_name is a field_name from the primary export list the function will
# use the grid from this field as the grid of the new maks we create
# lat_min,lat_max, lon_min, and lon_max are the corners of the box
# a real number (see mapl_parser for legal syntax for this). Inside the box
# the mask will be this number. If it is set to "undef" then the value will
# be MAPL_Undef
# a real number, same as above but mask will be this number outside of box
DerivedExports::
field_name expression refresh_template
::
\end{verbatim}

Currently for the primary exports, the last 3 keywords in each entry are optional, however all 3 must be specified if used.

The following is a description of a primary export entry:
\begin{trivlist}
\item[\tt field\_name]        Name of the field to be filled. This should be the same 
                              name as the field in the Import state {\tt MAPL\_ExtDataGridCompMod}
                              is attempting to fill.
\item[\tt Units]              Units - this is a place holder for the time being and while text must
                              be there is not used
\item[\tt field\_dims]        Field dimension - 'xyz' if 3D and 'xy' if 2D
\item[\tt level\_type]        'c' if MAPL\_VLocationCenter and 'e' if MAPL\_VlocationEdge 'NA' if 2D
\item[\tt climatology]        'Y' if variable represents climatological data. 'N' if not.
\item[\tt refresh\_template]  Refresh template. See section on the refresh template below for more details and function.
\item[\tt shift]              number to shift data, currently the data is always shifted so enter 0.0 if you want no shifting
\item[\tt scale]              number to scale data, currently the data is always scaled so enter 1.0 if you want no scaling
\item[\tt name\_on\_file]     name of the variable on the file
\item[\tt FileTemplate]       The full path to the file. The actual filename can be the real file name or a grads
                              style template. In addition you can simply set the import to a constant by specifying the 
                              entry as /dev/null:real\_constant. if no constant is specified after /dev/null with the colon
                              the import is set to zero.
\item[\tt reference\_time]    Reference time in the form YYYY-MM-DDThh:mm:ss
\item[\tt frequency\_units]   Units of the frequency of the primary export. Valid entries are years, months, days, hours, minutes
\item[\tt frequency]          An integer frequency of the file for this primary export.
\end{trivlist}

The following are descriptions of a mask entry:
\begin{trivlist}
\item[\tt mask\_name]         Name of the mask
\item[\tt mask\_function]     The function used to generate the mask. 
                              Currently the only supported function is the bounding box which is specified as
                              bbox(field\_name,lat\_min,lat\_max,lon\_min,lon\_max,real\_in,real\_out).
                              The arguements are as follows: field\_name is an input field whose grid
                              will be used to create the mask. The next four arguments define the corners
                              of a rectangular bounding box by specifiy the maximum latitudes and longitudes.
                              Finally real\_in and real\_out are numbers specify what the value of the mask
                              will be inside and outside of the box. In addition to specifying a real number
                              the user may specify "UNDEF" which will make the value MAPL\_UNDEF
\end{trivlist}

The following is a description of a derived export entry:
\begin{trivlist}
\item[\tt field\_name]        Name of the field to be filled
\item[\tt expression]         Legal expression understood by the MAPL parser, for more information see details
                              on legal expressions in the parser documentation. Legal variable names
                              are any field\_name in the primary export list or mask\_name in
                              the mask list. Remember that the fields in an expression
                              must have the same grid, otherwise an error will result.
\item[\tt refresh\_template]  Same options as with the primary export
\end{trivlist}

The Refresh Template controls when the data in the field being satisfied by Extdata is updated and currently has three options:

The first option is set the refresh template to '-'. In this case the field will only be updated once, the first time
the run method of {\tt MAPL\_ExtDataGridCompMod} is called. The data will be read from the file
obtained from evaluating the file template at the current model time.

The second option is to provide a template of the form \%y4-\%m2-\%d2T\%h2:\%n2:00 where
y4, m2, d2, h2, and n2 are the year,month, day, time, and minute. The user then
substitutes the time they wish the variable to be periodically updated. This is best illustrated by examples.
For example to update the variable daily at 21Z the refresh template should read
\%y4-\%m2-\%d2T21:00:00. At 21Z every day {\tt MAPL\_ExtDataGridCompMod} will attempt to read a
variable from the file obtained by apply the current time to the file template.
To update the variable monthly on the 1st of the month at 21Z the refresh
template should read \%y4-\%m2-01T21:00:00.

Finally the refresh\_template can be set to '0'. In this case the field will be updated
at every time step in the model run. Simply because the field is being updated at every timestep does not mean
there has to be data on file for each timestep. If you have the data on file at a larger frequency than the model
timestep ExtData can interpolate between the times for which data exists, even if that data is distributed among multiple files.

In order to understand how to use Extdata when the refresh\_template is '0' it is helpful to understand what it does it in this case.
The basic idea is that for a given model time ExtData tries to find two times, the
left and right bracketing time, on the available files that bracket the current time.
It then reads the data at those times and performs a linear interpolation to get the data at the current time.
It stores the bracketing data for later times to minimize the IO and updates the bracketing data as neccessary when the current time passes the
right bracketing time.
The data need not reside all on one file.

To get the data Extdata needs to the know what files are available and where to look for the data.
Extdata accomplishes this by assuming
that the files containing the data are timestamped starting at a reference time at some frequency when the
refresh\_template is '0'.
For example you could have a daily file, a monthly file, or a file every two hours starting at 13:30 on a particular day,
as long as there is data on the files that covers the time you want to run you application and that any times on the file
fall within the range of the timestamp. Thus if you have a file with the template myfile.\%y4\%m2\%d2.nc4, one particular
file might be myfile.20010414.nc4 and any times in this file must be sometime on the 14th of march 2001.

There are two ways to specify the reference time and frequency.
The first is to use the three optional keywords for a primary export. As described in the section on primary exports
you specify a reference time, and a frequency with some units. When ExtData tries to find the bracketing data it
starts by finding a file with a timestamp closest to but less than or equal to the current time. If it does not find a bracketing time
there it checks the file with the timestamp at the frequency interval before and after the file it just checked.

The second is to let ExtData figure out the frequency from the file template itself and leaving the final three entries in the
primary export blank. In that case it determines the rightmost token in the grads template and uses that as the frequency.
For example if the template is filename.\%y4\%m2\%d2.nc4 the frequency will be assumed to be days. The reference time is taken
to be the application start time at the beginning with any time units less than the frequency assumed to be zero. So using
the same file template again if we start on 2001-04-14 at 21z the reference time will be set to 2001-04-14 at 0z.

To hopefully make this clearer here are several concrete examples showing how it might be used.

Suppose we have data on daily files and each file has data at 0Z, 6Z, 12Z, and 18Z.
The file template is myfile.\%y4\%m2\%d2.nc4 and we leave the reference time and frequency blank.

\begin{verbatim}
if the clock starts at 21Z on January 1, 2001 these are the initial bracketing
times data is read from

myfile.20010101.nc4
0Z
6Z
12Z
18Z    Left bracket time
myfile.20010102.nc4
0Z     Right bracket time
6Z
12Z
18Z

When the time passes 0Z on 2nd of month we must update bracket time
and now the new bracketing times, for example
when it is 2Z on 2nd of month the data for the following times will be in memory

myfile.20010101.nc4
0Z
6Z
12Z
18Z 
myfile.20010102.nc4
0Z     Left bracket time
6Z     Right bracket time
12Z
18Z
\end{verbatim}

Now suppose we have data on files every two hours with one time per file, the time on file
being the same as the timestamp on the file. The first file is at 2001-01-01 at 01:30:00
Then the reference time is 2001-01-01T01:30:00, with a frequency of 2 hours.
The file template must be of the form myfile.\%y4\%m2\%d2\_\%h2\%n2z.nc4

\begin{verbatim}
if the clock starts at 2Z on January 1, 2001 and we have a series of files.
The initial bracket times are:

myfile.20010101_0130.nc4 Left bracket Time
myfile.20010101_0330.nc4 Right bracket Time
myfile.20010101_0530.nc4

Now when it is 4Z the bracket times are

myfile.20010101_0130.nc4
myfile.20010101_0330.nc4 Left bracket Time
myfile.20010101_0530.nc4 Right bracket Time

\end{verbatim}

Now suppose we have monthly climatologies. The file template is of the form
myfile.\%y4\%m2.nc4
Each file has data for one time, at noon on the 15th of the month. We leave the
reference time and frequency blank.

\begin{verbatim}
if the clock starts at 2001-04-01 0z then initial bracket times and data stored are 
myfile.200103.nc4
2001-03-15T12:00:00 left bracket time
myfile.200104.nc4
2001-04-15T12:00:00 right bracket time

later then time is 2001-04-17 0z then the bracket times and data stored are
myfile.200104.nc4
2001-04-15T12:00:00 left bracket time
myfile.200105.nc4
2001-05-15T12:00:00 right bracket time

\end{verbatim}
