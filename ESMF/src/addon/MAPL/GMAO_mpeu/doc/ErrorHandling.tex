% +-======-+ 
%  Copyright (c) 2003-2007 United States Government as represented by 
%  the Admistrator of the National Aeronautics and Space Administration.  
%  All Rights Reserved.
%  
%  THIS OPEN  SOURCE  AGREEMENT  ("AGREEMENT") DEFINES  THE  RIGHTS  OF USE,
%  REPRODUCTION,  DISTRIBUTION,  MODIFICATION AND REDISTRIBUTION OF CERTAIN 
%  COMPUTER SOFTWARE ORIGINALLY RELEASED BY THE UNITED STATES GOVERNMENT AS 
%  REPRESENTED BY THE GOVERNMENT AGENCY LISTED BELOW ("GOVERNMENT AGENCY").  
%  THE UNITED STATES GOVERNMENT, AS REPRESENTED BY GOVERNMENT AGENCY, IS AN 
%  INTENDED  THIRD-PARTY  BENEFICIARY  OF  ALL  SUBSEQUENT DISTRIBUTIONS OR 
%  REDISTRIBUTIONS  OF THE  SUBJECT  SOFTWARE.  ANYONE WHO USES, REPRODUCES, 
%  DISTRIBUTES, MODIFIES  OR REDISTRIBUTES THE SUBJECT SOFTWARE, AS DEFINED 
%  HEREIN, OR ANY PART THEREOF,  IS,  BY THAT ACTION, ACCEPTING IN FULL THE 
%  RESPONSIBILITIES AND OBLIGATIONS CONTAINED IN THIS AGREEMENT.
%  
%  Government Agency: National Aeronautics and Space Administration
%  Government Agency Original Software Designation: GSC-15354-1
%  Government Agency Original Software Title:  GEOS-5 GCM Modeling Software
%  User Registration Requested.  Please Visit http://opensource.gsfc.nasa.gov
%  Government Agency Point of Contact for Original Software:  
%  			Dale Hithon, SRA Assistant, (301) 286-2691
%  
% +-======-+ 
% 
\section{Error Handling}
%
The module {\tt m\_die} handles the errors that occur in the code.
When a problem is encountered, an error message is printed out and 
the code stops running.
The main procedures of this modules are:

\noindent {\tt die:} Signals an exception and uses {\tt perr} to point to
the location and the reason of the exception.

\noindent {\tt perr:} Sends a simple error message to the standard error
output.

\noindent {\tt MP\_die:} A special die() for MPI errors.

\noindent {\tt MP\_perr:} {\tt perr} for MPI errors, from the module
   {\tt m\_mpif90}.
%
\begin{verbatim}
EXAMPLE 1:
  Assume that we want to read input data from a file and for whatever
  reason (wrong format, missing data, etc.) the reading operation fails.

  Code:
      use m_die,only : die
      character(len=*) :: myname
      integer          :: rc
      myname = "name_of_my_subroutine"
      
      call reading_file(other_arguments, rc)
      if (rc.ne.0) then
         call die(myname, 'could not read input file')
      endif

  Output:
      name_of_my_subroutine: could not read input file

EXAMPLE 2:
  Assume that we read from a file a certain number of observations
  (size of the variable ``rlat''). The code is expecting a number of
  observations at least equal to ``nobs''.

  Code:
      use m_die,only : perr
      character(len*) :: myname
      integer
      myname = ``name_of_my_subroutine''

      if (size(rlat)<nobs) &
         call perr(myname,'nobs',nobs,'size(rlat)',size(rlat))

  Output:
      name_of_my_subroutine: error, nobs=****, size(rlat)=$$$$

EXAMPLE 3:
  Here we describe the use of MP_die. We assume that we want to
  broadcast a variable and a problem occurs.
 
  Code:
      use m_die,only : MP_die
      use m_mpif90,only : MP_type
      character(len=*) :: myname
      real             :: rbcast(20)
      integer          :: ier
      integer          :: root, comm
      myname = ``name_of_my_subroutine''
      
      call MPI_bcast(rbcast,size(rbcast),MP_type(rbcast),root,comm,ier)
      if(ier/=0) call MP_die(myname,'MPI_bcast()',ier)

  Output:
      name_of_my_subroutine: MPI_bcast(), error, ierror=ier

\end{verbatim}
 
