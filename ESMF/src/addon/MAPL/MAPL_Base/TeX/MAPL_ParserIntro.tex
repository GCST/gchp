% +-======-+ 
%  Copyright (c) 2003-2007 United States Government as represented by 
%  the Admistrator of the National Aeronautics and Space Administration.  
%  All Rights Reserved.
%  
%  THIS OPEN  SOURCE  AGREEMENT  ("AGREEMENT") DEFINES  THE  RIGHTS  OF USE,
%  REPRODUCTION,  DISTRIBUTION,  MODIFICATION AND REDISTRIBUTION OF CERTAIN 
%  COMPUTER SOFTWARE ORIGINALLY RELEASED BY THE UNITED STATES GOVERNMENT AS 
%  REPRESENTED BY THE GOVERNMENT AGENCY LISTED BELOW ("GOVERNMENT AGENCY").  
%  THE UNITED STATES GOVERNMENT, AS REPRESENTED BY GOVERNMENT AGENCY, IS AN 
%  INTENDED  THIRD-PARTY  BENEFICIARY  OF  ALL  SUBSEQUENT DISTRIBUTIONS OR 
%  REDISTRIBUTIONS  OF THE  SUBJECT  SOFTWARE.  ANYONE WHO USES, REPRODUCES, 
%  DISTRIBUTES, MODIFIES  OR REDISTRIBUTES THE SUBJECT SOFTWARE, AS DEFINED 
%  HEREIN, OR ANY PART THEREOF,  IS,  BY THAT ACTION, ACCEPTING IN FULL THE 
%  RESPONSIBILITIES AND OBLIGATIONS CONTAINED IN THIS AGREEMENT.
%  
%  Government Agency: National Aeronautics and Space Administration
%  Government Agency Original Software Designation: GSC-15354-1
%  Government Agency Original Software Title:  GEOS-5 GCM Modeling Software
%  User Registration Requested.  Please Visit http://opensource.gsfc.nasa.gov
%  Government Agency Point of Contact for Original Software:  
%  			Dale Hithon, SRA Assistant, (301) 286-2691
%  
% +-======-+ 
{\tt MAPL\_NewArthParserMod} is a module that provides a mathematical
parsing capabiltiy to MAPL and ESMF components. The module evaluates an ESMF field element by element
using an expression which could contain other ESMF fields as variables.
Examples of use for this component would be to take the Log of every element in a field
or add two fields element by element.
The heart of the module is
a simple public routine MAPL\_StateEval(state,expression,field,rc) with three required arguments.
The arguments are as follows: \newline
state is an \texttt{ESMF\_State} type containing some number of fields.\newline
expression is a character string containing a valid mathematical function string.\newline
field is an \texttt{ESMF\_Field} type that will be filled using the expression.\newline

The expression can contain the names of any fields in the state as variables.

The following can appear in the expression string\\
1. The function string can contain the following mathematical operators +, -, *, /, \^{} and ()\\
2. Variable names - these can be any field name in the input state. Parsing of variable names is case sensitive.\\
3. The following single argument fortran intrinsic functions and user defined functions are implmented:
exp, log10, log, sqrt, sinh, cosh, tanh, sin, cos, tan, asin,
acos, atan, heav (the Heaviside step function).
Parsing of functions is case insensitive.\\
4. Integers or real constants.
To be recognized
as explicit constants these must conform to the format

[+\textbar-][nnn][.nnn][e\textbar E\textbar d\textbar D[+\textbar-]nnn]

where nnn means any number of digits. The mantissa must contain at least
one digit before or following an optional decimal point. Valid exponent
identifiers are 'e', 'E', 'd' or 'D'. If they appear they must be followed
by a valid exponent!

Operations are evaluated in the order
\begin{enumerate}
\item ()      expressions in brackets
\item -X      unary minux
\item X\^{}Y  exponentiation
\item X*Y X/Y multiplicaiton and division
\item A+B X-Y addition and subtraction
\end{enumerate}

One logical requirement is that the fields in the state and the field being filled are on the same grid,
including vertical levels. For example, 3D fields in an expression must ALL have been created with
the same vertical levels (MAPL\_DimsVLocationEdge or MAPL\_DimsVLocationCenter). If not an error will result.
The one exception is operations involving 3D and 2D fields when
the resultant field is a 3D field. In this case, the operation is performed between each level
of the 3D field and the 2D field. This is useful if one wanted to scale each level of a 3D
field with the same 2D field. Of course the first two dimensions of the 3D field must be same as the 2D field.

The parser also obeys undef arithmetic. Any arithemtic operation involving MAPL\_Undef or function of MAPL\_Undef
results in MAPL\_Undef.

The following are several examples of valid expressions. For the examples we will assume that the input state
has 4 fields A, B, C, and D.\newline
1. B*2.0e0\newline
2. sqrt(A*A+B*B)\newline
3. A*heav(B)\newline
4. A\^{}(C+D)-2.0e-3
