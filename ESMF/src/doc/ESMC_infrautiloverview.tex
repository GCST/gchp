% $Id$
%
% Earth System Modeling Framework
% Copyright 2002-2019, University Corporation for Atmospheric Research, 
% Massachusetts Institute of Technology, Geophysical Fluid Dynamics 
% Laboratory, University of Michigan, National Centers for Environmental 
% Prediction, Los Alamos National Laboratory, Argonne National Laboratory, 
% NASA Goddard Space Flight Center.
% Licensed under the University of Illinois-NCSA License.

%TODO: This file started as an exact copy of the Fortran version of this file.
%TODO: Changes were made to correctly reflect the current status of the C API.
%TODO: Eventually this file should be removed again and replaced by a single
%TODO: generic version that can be included for both Fortran and C refdocs.

\section{Overview of Infrastructure Utility Classes}

The ESMF utilities are a set of tools for quickly assembling modeling applications.

The Time Management Library provides utilities for time and time interval representation, as well as a higher-level utility, a clock, that controls model time stepping.

The ESMF Config class provides configuration management based on NASA DAO's Inpak package, a collection of methods for accessing files containing input parameters stored in an ASCII format.

The ESMF LogErr class consists of a method for writing error, warning, and informational messages to a default Log file that is created during ESMF initialization.

The ESMF VM (Virtual Machine) class provides methods for querying information about a VM. A VM is a generic representation of hardware and system software resources. There is exactly one VM object per ESMF Component, providing the execution environment for the Component code. The VM class handles all resource management tasks for the Component class and provides a description of the underlying configuration of the compute resources used by a Component.  In addition to resource description and management, the VM class offers the lowest level of ESMF communication methods.
